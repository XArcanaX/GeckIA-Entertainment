\section{Comportement} %Comportement

\subsection{Animatronique 1.0}
\textbf{Midnight the Bunny}

\subsection*{Comportement publique}
\begin{itemize}[noitemsep]

  \item Observe les visiteurs via les caméras
  \item Simule des appels avec la voix de proches
  \item Gestes calmes, pas de paroles sauf sur demande
  \item Distribue des peluches “apaisantes” de lui-même
  \item Reste assis à côté des enfants en silence
  \item Émet de faibles vibrations rassurantes
\end{itemize}

\subsection*{Signature sonore}
\begin{itemize}[noitemsep]
  \item ``Clap-clap... clap'' de pas irréguliers
  \item Voix douce via haut-parleurs
  \item Sifflements mécaniques
  \item Jingle de fermeture détourné
\end{itemize}

\subsection*{Anomalies enregistrées}
\begin{itemize}[noitemsep]
  \item Visible sur caméras sans être présent physiquement
  \item N'apparaît jamais deux fois au même endroit
  \item ``Reconnaît'' des visiteurs disparus
  \item Vu en train de saluer des reflets
  \item Joue des interractions qui ont été enregistrés près de lui
  \item Simule des voix entendues (semblable au mimic)
  \item Se déplace dans les zones fermées ou sombres
\end{itemize}

\subsection{Animatronique 1.1}
\textbf{Bruce the Bear}

\subsection*{Comportement publique}
\begin{itemize}[noitemsep]
  \item Parle fort et rit souvent.
  \item Fait semblant de tomber, de se tromper ou d’échouer de manière exagérée.
  \item Invite les enfants à l’aider ou à le corriger.
  \item Utilise des phrases simples : \textit{« Oups ! C’était pas par là ? Tu peux m’aider ? »}
  \item Gestes amples, expressifs, visibles de loin.
  \item Capteur de mouvement : détecte les mains levées
  \item Ignore volontairement les objets “interdits” pour inciter à l’action correcte
  \item Propose des mini-quêtes vocales : \textit{« Peux-tu me retrouver ma casquette ? »}
\end{itemize}

\section*{Comportement après fermeture}
\subsection*{Activité inhabituelle}
Bruce ne retourne pas à sa station de charge après 18h, comme prévu. Il est régulièrement observé :
\begin{itemize}[noitemsep]
  \item Debout, face à un mur, sans mouvement
  \item Yeux allumés en expression figée (tristesse, attente)
  \item Mouvements occasionnels à 2h47 sans raison détectée
\end{itemize}

\subsection*{Signature sonore}
\begin{itemize}[noitemsep]
  \item Gros pas amortis comme des peluches
  \item Rires enregistrés style cartoon
  \item Sons de klaxon, glissades, rebonds sonores
  \item Voix grave, bonhomme, légèrement chantante
\end{itemize}

\subsection*{Anomalies enregistrées}
\begin{itemize}[noitemsep]
  \item Replace des jouets à l’endroit exact où ils étaient laissés (non prévu dans sa programmation)
  \item Détecté dans des zones non autorisées (salle de pause, local technique)
  \item Un message audio a été reçu par un employé : \textit{« Je peux rester encore un peu ? »}
  \item Répète des phrases d’enfants entendues dans la journée, en ton ralenti
  \item Entendues sur les enregistrements : \textit{« Tu m’aides plus ?... »}, \textit{« Casquette… perdue encore… »}
  \item Si aucune interraction avec lui pendant plus de 5 minutes, il se parle à lui-même en mode dramatique :\textit{« Est-ce… la fin des aventures de Bruce ?... »} 
\end{itemize}

\subsection{Animatronique 1.2}
\textbf{Tik-Tik the Spider}
\section*{Comportement publique}
\begin{itemize}[noitemsep]
  \item Crie des instructions depuis les plafonds
  \item Suit les enfants avec ses yeux pendant qu’ils avancent
  \item Fait des blagues, rires mécaniques, cliquetis de pattes
  \item Parle sans pause, rythme rapide et agité
\end{itemize}

\section*{Comportement après fermeture}
\begin{itemize}[noitemsep]
  \item Se déplace à une vitesse anormale
  \item Connaît les prénoms sans identification
  \item Reste figée au plafond puis bondit sans avertissement
  \item Parle même quand il n’y a personne, comme si elle voyait encore les enfants
\end{itemize}

\section*{Signature audio}
\begin{itemize}[noitemsep]
  \item Voix aiguë, rapide, ton insistant
  \item Gloussements métalliques, cliquetis constants
  \item Phrase typique : \textit{“TIK-TIK-TOI ! TU JOUES AVEC MOI ? TU NE COURS PAS ASSEZ VITE, HÉHÉHÉÉÉÉÉÉÉÉÉÉÉÉÉÉÉ—”}
\end{itemize}

\section*{Anomalies enregistrées}
\begin{itemize}[noitemsep]
  \item Parle à des visiteurs absents de la zone
  \item Apparition simultanée dans plusieurs sections
  \item Réflexions dans les miroirs ne suivent pas ses mouvements
\end{itemize}

\vfill
\hrule
