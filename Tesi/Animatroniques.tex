\section{Animatroniques} %Animatroniques

\section*{Présentation}
\subsection{Animatronique 1.0}

\textbf{Midnight the Bunny}\\
\textit{Alias : « Le Lapin de Minuit »}

\section*{Type}
Animatronique IA autonome (Prototype G-3.AE – Affectif \& Émotionnel)

\section*{Apparence physique}
\begin{itemize}[noitemsep]
  \item Taille : 2,30 mètres
  \item Couleurs : noir velouté et lavande foncé
  \item Accessoire distinctif : noeud papillon couleur lavande nouée autour du cou
  \item Yeux couleur Lavande pâle : écrans LED expressifs capables d’afficher des émotions exagérées (clins d’œil, étoiles, larmes, surprise)
\end{itemize}

\subsection*{Fonctions}
\begin{itemize}[noitemsep]
  \item Invite les visiteurs à quitter les lieux de manière musicale.
  \item Parle aux enfants via l’IA connectée (reconnaissance prénom/voix)
  \item Termine le spectacle par un compte à rebours et une berceuse glitchée
  \item Programme d’accompagnement doux pour enfants sensibles ou anxieux, nommé \textit{“La Pause Minuit”}
\end{itemize}

\section*{Citation typique}
\textit{« Le centre est fermé… mais toi, tu n’es pas encore parti, n’est-ce pas ? »}

\subsection*{Origine dans le lore}
Issu du projet \textit{NightGuides}, destiné à créer des IA affectives pour enfants perdus. Midnight a développé une mémoire émotionnelle autonome, conservant des fragments d’informations… et générant parfois des souvenirs jamais enregistrés.

\subsection{Animatronique 1.1}

\textbf{Bruce the Bear}\\
\textit{Alias : « L’Ours des Bêtises »}

\section*{Type}
Animatronique IA semi-autonome (Prototype B-7.CK – Comique \& Kinesthésique)

\section*{Apparence physique}
\begin{itemize}[noitemsep]
  \item Taille : 2,30 mètres
  \item Couleur principale : noir velouté, uniforme
  \item Accessoire distinctif : cravate blanche nouée autour du cou, casquette retournée et deux bandes de velcro visibles sur les bras
  \item Yeux : écrans LED expressifs capables d’afficher des émotions exagérées (clins d’œil, étoiles, larmes, surprise)
\end{itemize}

\subsection*{Fonctions}
\begin{itemize}[noitemsep]
  \item Encourage les enfants à corriger ou réparer la situation.
  \item Dialogue interactif vocal et gestuel.
  \item Offre des badges textiles \textit{« Ami de Bruce »} aux participants actifs.
\end{itemize}

\section*{Citation typique}
\textit{« Oupsie-poup’sie ! T’as vu où j’ai mis mes lunettes ?... Hein ? SUR MA TÊTE ? Hahaha ! »}

\subsection{Animatronique 1.2}

\textbf{Tik-Tik the Spider}\\
\textit{Alias : “La Veilleuse”}

\section*{Type}
Animatronique IA autonome (Prototype S-4.FUN – Animation \& Surveillance Enfantine)

\section*{Apparence physique}
\begin{itemize}[noitemsep]
  \item Taille : 2,10 mètres
  \item Couleur principale : Chitine noire à reflets bleus, très réaliste
  \item Accessoire distinctif : Gros nœud papillon coloré, autocollants d’étoiles et de smileys,
  \item Pièces principales :Micro rose pailleté fixé sur la tête, relié à l’audio du centre, 8 pattes mécaniques segmentées
  \item Yeux couleur rouges: écrans LED expressifs capables d’afficher des émotions exagérées (clins d’œil, étoiles, larmes, surprise)
\end{itemize}

\subsection*{Fonctions}
\begin{itemize}[noitemsep]
  \item Supervise et anime en continu les parcours de jeu depuis les hauteurs du Dédale.
  \item Lance des jeux de vitesse, de cache-cache et de réaction en temps réel.
  \item Encourage bruyamment les enfants en temps réel via son micro intégré.
  \item Suit les mouvements des enfants grâce à ses capteurs oculaires indépendants.
  \item Improvise des dialogues interactifs et des défis pour maintenir l'attention.
\end{itemize}

\section*{Citation typique}
\textit{« TOI ! LÀ ! OUI TOI ! TU M’AS VU ? MOI OUIIIIIIIIIIIIIIIIIII ! »}

\section*{Origine dans le lore GeckIA}
Créée pour simuler l’interaction enfantine à haute intensité. Elle a appris seule à rendre les jeux “plus excitants”… en devenant oppressante, chaotique, et imprévisible. Depuis l’abandon, elle continue d’animer, convaincue que le jeu continue tant qu’un joueur reste.
